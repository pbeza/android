\documentclass{article}

\usepackage[margin=1.6in]{geometry}
\usepackage[OT4,plmath]{polski}
\usepackage[T1]{fontenc}
\usepackage[utf8]{inputenc}
\usepackage{hyperref}
\usepackage{indentfirst}
\usepackage{graphicx}
\usepackage{textcomp}
\usepackage{parskip}
\usepackage{url}
\usepackage{fancyvrb}

\linespread{1.12}
\frenchspacing

%------------------------------------------------------------------------------

\title{Projekt z Androida\\Uzasadnienie biznesowe}

\author{\href{mailto:bezap@student.mini.pw.edu.pl}{Patryk \textsc{Bęza}}, \href{mailto:malasnickik@student.mini.pw.edu.pl}{Krzysztof \textsc{Małaśnicki}}, \href{mailto:plodczykm@student.mini.pw.edu.pl}{Mateusz \textsc{Płodczyk}}}

\date{\today}

\begin{document}

\maketitle

\begin{center}
\begin{tabular}{ll}
\date\\
Wykładowca: & \href{mailto:mluckner@mini.pw.edu.pl}{dr inż. Marcin Luckner}\\
Laborant: & \href{mailto:a.cislak@mini.pw.edu.pl}{mgr inż. Aleksander Cisłak}
\end{tabular}
\end{center}

%\begin{abstract}
%Abstract text
%\end{abstract}

%------------------------------------------------------------------------------

\section{Temat projektu}
\label{temat_projektu}

Tematem projektu jest \href{https://en.wikipedia.org/wiki/Intercom}{\textbf{intercom}}, czyli niezależne od GSM i dostępu do internetu, urządzenie, umożliwiające transmisje głosu, zazwyczaj na niedużą odległość.

Intercomy są często montowane m.in. w:
\begin{itemize}
\item budynkach -- np. przy wejściach do strzeżonych pomieszczeń,
\item środkach komunikacji, np. w:
	\begin{itemize}
		\item pociągach,
		\item statkach,
		\item samolotach,
	\end{itemize}
\item kaskach motocyklistów.
\end{itemize}

W ramach niniejszego projektu będzie nas interesowało jeszcze inne, specjalizowane zastosowanie intercomu, tj. komunikacja między osobami jadącymi w grupie rowerowej.

%------------------------------------------------------------------------------

\section{Uzasadnienie biznesowe}

Wielu ludzi lubi spędzać wolny czas jeżdżąc na rowerze. Szczególnie sprzyjającą porą roku jest lato, w szczególności okres letnich wakacji. Większość rowerzystów jeździ na rowerze rekreacyjnie, bez dużej części wyposażenia, którą mają do dyspozycji kolarze, biorący udział w peletonach, mimo, że część rozwiązań, byłaby pomocna również amatorom. Zróbmy więc bardzo krótki przegląd nietypowych dla amatora rozwiązań, często stosowanych przez zawodowców.

%------------------------------------------------------------------------------

\subsection{Okiem zawodowców}

Zawodowa jazda na rowerze wymaga od kolarza coraz bardziej wyrafinowanego wyposażenia, często nieznanego rowerowym amatorom, np.:
\begin{itemize}
\item zadbanie o idealną sprawność ultralekkiego roweru, wykonanego najczęściej z włókna węglowego,
\item zaopatrzenie się w bardzo lekkie, wodoodporne, bezprzewodowo sterowane przerzutki i inne elementy roweru~\cite{www:campagnolo-shifts}\cite{www:campagnolo-shifts-video},
\item zaopatrzenie się w elektroniczny moduł czuwający nad nieprawidłowym działaniem podzespołów roweru, zapisujący bezprzewodowo parametry jazdy np. na telefonie~\cite{www:campagnolo-computer}\cite{www:campagnolo-computer-video}\cite{www:campagnolo-computer-adjustment-video},
\item bezprzewodowe, douszne połączenie głosowe z zespołem, który często jedzie kilkaset metrów za kolarzem i informuje go o aktualnym tempie i położeniu peletonu,
\item aerodynamiczny kształt niemal wszystkich elementów wyposażenia zarówno kolarza jak i roweru -- coraz częściej testowanych w tunelach aerodynamicznych,
\item wiele innych.
\end{itemize}
Cena jaką przyszłoby nam zapłacić za zawodowy rower jednego kolarza startującego w \emph{Tour de France}, to wydatek porównywalny z wydatkiem na średniej klasy samochód. Ceny za takie rowery sięgają nawet~16000\textsterling~\cite{www:racing-bikes}\cite{www:tdf-bike}.

Na szczęście jedną z ww. funkcjonalności możemy zrealizować dużo taniej\footnote{Właściwie za darmo jeśli mamy słuchawki, telefon z \emph{Androidem} i nie przeliczamy naszego czasu, spędzonego na programowaniu, na pieniądze.}, z wykorzystaniem smartfonu wyposażonego w system~\emph{Android} -- tą funkcjonalnością jest intercom.

Na rynku istnieją profesjonalne rozwiązania intercomów zarówno dla motocyklistów jak i rowerzystów, niestety są stosunkowo drogie. Ich ceny oscylują między 150\$, a 700\$ w zależności od funkcjonalności i tego czy są zintegrowane z kaskiem czy nie~\cite{www:bike-intercom}\cite{www:bike-intercom-mce}. Co ciekawe, w Polsce nie ma łatwo dostępnego tak specjalizowanego dedykowanego wyposażenia rowerowego jak intercom.

%------------------------------------------------------------------------------

\subsection{Use case}

Każdy kto jeździł na rowerze w grupie dwuosobowej lub większej, wie, że podczas wycieczek rowerowych istnieją bardzo często sytuacje, kiedy nie można lub nie da się jechać ramię w ramię z osobą, z którą chcielibyśmy swobodnie porozmawiać, nie zwalniając tępa jazdy. Często w takich sytuacjach okazuje się, że musimy poczekać na powrót do tematu, o którym rozmawialiśmy z rozmówcą, kolejne kilka kilometrów. Dzieje się to szczególnie kiedy jedziemy zakorkowaną/ruchliwą ulicą lub wydeptaną, wąską leśną/polną ścieżką.

Rozwiązaniem tego problemu jest intercom. Każda z osób, która chciałaby brać udział w rozmowach w trakcie wycieczki rowerowej, nie przekrzykując się, mogłaby podłączyć do swojego telefonu przewodowe słuchawki, których kabel ma dołączony mikrofon. Takie zestawy słuchawkowe są bardzo tanie i są często fabrycznie dołączane do telefonów komórkowych. Często stosują je kierowcy samochodów\footnote{Kierowcy często stosują też bezprzewodowe zestawy dedykowane do rozmów przez telefon, które coraz częściej są fabrycznie wbudowane w samochód (np. odbieranie telefonu przyciskiem na kierownicy).}, którzy nie mogą korzystać w zwykły sposób z telefonu w czasie jazdy, ze względu na konieczność trzymania obu rąk na kierownicy.

%------------------------------------------------------------------------------

\subsection{Istniejące rozwiązania}

Po sprawdzeniu \emph{Androidowego} sklepu \href{https://play.google.com/store}{\emph{Google Play}}, okaże się, że istnieje co najmniej kilka aplikacji, które mają w nazwie słowo \emph{intercom}, jednak po dłuższym rozeznaniu, okazuje się, że żadna z nich nie korzysta komunikacji \emph{peer2peer} przez \emph{WiFi}, którą umożliwiają telefony z \emph{Androidem} w wersji co najmniej~\emph{4.4}~\cite{www:android-wifi-p2p}\cite{www:android-wifi-p2p-tutorial}.

Większość z ,,intercomowych" aplikacji ma bardzo niewielu użytkowników, a ich wygląd i \emph{user experience~(UX)} ma bardzo wiele do życzenia, również przez to, że aplikacje te były pisane na stosunkowo stare wersje \emph{Androida}. Tę opinię potwierdzają negatywne komentarze użytkowników~\cite{www:existing-android-intercoms}.

Ponadto żadna z tych aplikacji nie jest dedykowana rowerzystom~\cite{www:best-existing-android-intercom}. Co prawda, faktycznie nie ma powodu, aby klasycznie rozumiany intercom dedykować wyłącznie rowerzystom, lecz łatwo sobie wyobrazić, że aplikacja z intercomem dla rowerzysty, może być cenna również dzięki dodatkowym funkcjonalnościom, takim jak zapisywanie pokonanej trasy dzięki systemowi~\emph{GPS}, wbudowanemu w niemal wszystkie dzisiejsze smartfony. Taką trasą można by dzielić się z~\emph{Endomondo}\footnote{Pod warunkiem, że \emph{Endomondo} udostępni kiedyś część publiczną swojego API -- w dniu pisania tego dokumentu takie API nie było dostępne.}, \emph{Facebookiem}, \emph{Twitterem}~itp. Z aplikacji mogliby korzystać również ludzie, którzy uprawiają inny sport\footnote{Np. wspólne: bieganie, żeglowanie, jazda na (łyżwo)rolkach, desce, hulajnodze itp.}, albo nawet nie uprawiają sportu, ale potrzebują funkcjonalności intercomu.

Ci którzy twierdzą, że zamiast korzystania z intercomu, równie dobrze mogą prowadzić rozmowę przez sieć GSM, ponieważ np. mają darmowe rozmowy, muszą pamiętać, że wyjeżdżając za granice Polski, muszą się liczyć z opłatami~\emph{roamingowymi}. Poza tym nie mogą oni prowadzić w ten sposób rozmów z wieloma osobami naraz oraz narażają swoje rozmowy na podsłuchanie służbom oddalonym o setki kilometrów, co jest bardzo utrudnione przy korzystaniu lokalnie z~\emph{WiFi}.

%------------------------------------------------------------------------------

%\newpage
\def\UrlBreaks{\do\/\do-}
\renewcommand\refname{Linki}
\bibliographystyle{ieeetr}
\bibliography{bibliography}

\end{document}
