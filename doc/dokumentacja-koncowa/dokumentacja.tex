\documentclass[a4paper,titlepage]{article}
\usepackage[OT4,plmath]{polski}
%\usepackage[backend=biber]{biblatex}
\usepackage[margin=1in]{geometry}
\usepackage[noend]{algpseudocode}
\usepackage[page]{appendix}
\usepackage[usenames,dvipsnames,svgnames,table]{xcolor}
\usepackage[utf8]{inputenc}
\usepackage{adjustbox}
\usepackage{algorithm}
\usepackage{amsfonts}
\usepackage{amsmath}
\usepackage{amssymb}
\usepackage{amsthm}
\usepackage{array}
\usepackage{csquotes}
\usepackage{enumitem}
\usepackage{graphicx}
\usepackage{hyperref}
\usepackage{indentfirst}
\usepackage{longtable}
\usepackage{multirow}
\usepackage{parskip}
\usepackage{pifont}
\usepackage{setspace}
\usepackage{verbatimbox}
\usepackage{wrapfig}

\linespread{1.4}

%\addbibresource{Dokumentacja wstepna.bib}

\renewcommand*\appendixpagename{Załącznik}
\renewcommand{\qedsymbol}{$\square$}
\renewcommand{\algorithmiccomment}[1]{\hfill\textcolor{black!65}{\textit{#1}}}
\let\emptyset\varnothing

\makeatother
\setlength{\parindent}{24pt}
\theoremstyle{break}
\newtheorem*{uwaga}{Uwaga}
\newtheorem{definicja}{Definicja}[section]
\newtheorem{ozn}{Oznaczenie}[section]

\newcommand{\cmark}{\textcolor{ForestGreen}{\ding{51}}}
\newcommand{\xmark}{\textcolor{Maroon}{\ding{55}}}

%------------------------------------------------------------------------------

\title{Dokumentacja projektu}
\author{
    \href{mailto:bezap@student.mini.pw.edu.pl}{Patryk~Bęza}\\[0.7em]
    \href{mailto:plodczykm@student.mini.pw.edu.pl}{Mateusz~Płodczyk}\\[0.7em]
}
\date{\today}

\begin{document}

\makeatletter
\renewcommand{\ALG@name}{Algorytm}
\begin{titlepage}
\newcommand{\HRule}{\rule{\linewidth}{0.5mm}}
\center

\includegraphics[width=2.0cm]{img/mini}\\[1.5cm]
\textsc{\LARGE Politechnika Warszawska}\\[0.3cm]
\textsc{\Large Wydział Matematyki i~Nauk Informacyjnych}\\[1.5cm]
\textsc{\large Aplikacje mobilne: Android}\\[0.2cm]
\textsc{\small Rok akademicki 2015/2016}\\[1.5cm]

\HRule \\[1cm]
{ \huge \bfseries \@title}\\[0.7cm]
\HRule \\[1.75cm]

\begin{minipage}[t]{0.4\textwidth}
\begin{flushleft}\large
\textsc{Autorzy:}\\[3mm]
\@author
\end{flushleft}
\end{minipage}
\begin{minipage}[t]{0.4\textwidth}
\begin{flushright}\large
\textsc{Wykładowca:}\\[3mm]
\href{mailto:M.Luckner@mini.pw.edu.pl}{dr~inż.~Marcin~Luckner}\\[1cm]
\textsc{Laborant:}\\[3mm]
\href{mailto:A.Cislak@mini.pw.edu.pl}{mgr~inż.~Aleksander~Cisłak}\\[1cm]
\end{flushright}
\end{minipage}
\vfill
{\large \today}

\end{titlepage}

%\maketitle
\tableofcontents
\clearpage

%------------------------------------------------------------------------------

\begin{abstract}

\begin{figure}[t]
    \centering
    \includegraphics[width=0.3\textwidth]{img/intercom}
\end{figure}

Niniejszy dokument powstał w~ramach dokumentacji projektu zespołowego z~przedmiotu \textit{Aplikacje mobilne: Android} w semestrze letnim roku akademickiego~2015/2016 na \emph{\href{http://www.mini.pw.edu.pl/}{Wydziale MiNI}~\href{https://www.pw.edu.pl/}{Politechniki Warszawskiej}}. Ma on za zadanie udokumentować ideę działania aplikacji mobilnej, sposób jej użycia, wykorzystane narzędzia i biblioteki oraz wyniki testów funkcjonalnych i wydajnościowych oraz wnioski.

\end{abstract}

\clearpage

%------------------------------------------------------------------------------

\section{Wstęp}
\label{wstep}
\noindent W tym rozdziale przedstawiono krótki opis idei działania tworzonej aplikacji mobilnej.

\subsection{Cel aplikacji}

W ramach projektu powstanie aplikacja mobilna na system \href{http://www.android.com/}{Android}, która będzie miała na celu... \emph{TODO}.

%------------------------------------------------------------------------------

\subsection{Idea działania aplikacji}

Rozwinięcie celu aplikacji \emph{TODO}...

%------------------------------------------------------------------------------

\section{Aplikacja}

Do stworzenia aplikacji mobinej na system \emph{\href{http://www.android.com/}{Android}} zostanie wykorzystane środowisko programistyczne bazujące na \emph{\href{https://www.jetbrains.com/idea/}{IntelliJ~IDEA}} czeskiej firmy \emph{\href{https://www.jetbrains.com/}{JetBrains}}, tzn.~\emph{\href{http://developer.android.com/tools/studio/index.html}{Android Studio}}, które jest aktualnie oficjalnym \emph{IDE} dla programowania aplikacji dla~\emph{Androida}.

\subsection{Dane wejściowe}

\emph{TODO}

%------------------------------------------------------------------------------

\subsection{Dane wyjściowe}

\emph{TODO}

%------------------------------------------------------------------------------

\subsection{Instrukcja obsługi}

\emph{TODO}

%------------------------------------------------------------------------------

\section{Testy}

Ten rozdział opisuje wyniki przeprowadzonych testów funkcjonalnych i~wydajnościowych.

\subsection{Dane testowe}

\emph{TODO}

%------------------------------------------------------------------------------

\subsection{Wyniki testów}

\emph{TODO}

%------------------------------------------------------------------------------

\newpage

\begin{appendices}

\section{Podział prac}
\subsection{Podział pracy pisemnej}

Poniższa tabela przestawia wykaz rozdziałów opracowanych przez poszczególnych członków zespołu.

\begin{table}[H]
\center
\begin{tabular}{p{2.5cm}|p{5cm}|p{5cm}}
& Patryk Bęza & Mateusz Płodczyk \\\hline\hline
\parbox{3cm}{\ \\Opracowane \\rozdziały} & -- & --\\
\end{tabular}
\end{table}

\subsection{Podział implementacji}

Poniższa tabela przestawia podział prac implementacyjnych w ramach projektu.

\begin{table}[H]
\center
\begin{tabular}{p{2.5cm}|p{5cm}|p{5cm}}
& Patryk Bęza & Mateusz Płodczyk \\\hline\hline
\multirow{3}{*}{\parbox{3cm}{\ \\Opracowane \\funkcjonalności}} & $\bullet$ Audiotrack - wątki & $\bullet$ Notification\\
& $\bullet$ GUI & $\bullet$ Settings \\
& $\bullet$ p2p & $\bullet$ p2p \\
& $\bullet$ Service & $\bullet$ Service (poprawki) \\
\end{tabular}
\end{table}

\end{appendices}

%------------------------------------------------------------------------------

%\clearpage
%\printbibliography[title=Bibliografia]

\end{document}
